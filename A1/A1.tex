\documentclass[12pt,a4paper]{article}

\setlength{\headheight}{15pt}
\usepackage[margin=1in]{geometry}
\usepackage{fancyhdr}
\usepackage{amssymb}
\usepackage{xcolor}

\pagestyle{fancy}
\fancyhf{}
\renewcommand{\headrulewidth}{0pt}
\fancyhead[L]{COMP 350 - A1 - Adrien Bélanger - 261118168}

\begin{document}

\subsection*{Question 1}
So, sign is - because of the first bit. Then, we flip the bits, add 1 and we add up the power of twos to get the number in decimal representing the magnitude. Combine with the negative symbol, and we have our number!
\\ 
1. Flip the bits
$$(11111111111111111111111111101010)_2 = (0000000000000000000000000000010101)_2$$
2. Add 1
$$(000000000000000000000000000010101 + 1)_2 = (000000000000000000000000000010110)_2 $$
3. Calculate the magnitude 
$$2^4 + 2^2 + 2^1 = 22$$
4. Combine
$$-22$$
\newline
Thus, our answer is $-22$.
\vfill
\newpage

\subsection*{Question 2}

\subsubsection*{a.}
The exponent bias should be 7.



\subsubsection*{b.}
The machine espilon is 

$$(0 0111 001 - 0 0111 000)_2 = \epsilon = 2^{-3} = 0.125$$

\subsubsection*{c. }

Largest: $(0 1110 111)_2 = 2^7 * (1 + 2^{-1} + 2^{-2} + 2^{-3}) = 128 * 1.875 = 240$
\newline
Smallest: $(0 0001 000)_2 = 1 * 2^{-6} = 0.015625$



\subsubsection*{d. }
No, not all integers. Proof by counterexample that not any integer can be represented. For example, 239 cannot be stored.
We can store 240, as seen above. However, the smallest step under would be $(0 1110 110)_2$
which is $2^7 * (1 + 2^{-1} + 2^{-2}) = 128 * 1.75 = 224$.
\newline
Thus, since 239 is not representable, by contraditction we prove that not all ints are representable.




\subsubsection*{e. } 

largest: $(0 0000 111)_2 =2^{-6} * (2^{-1} + 2^{-2} + 2^{-3}) = 0.013671875 $
\newline
smallest: $(0 0000 001)_2 = 2^{-6} * 2^{-3} = 0.001953125$




\subsubsection*{f. }

5 is $101_2$ so exponent is 2+7 = 9 so $(1001)_2$. Then, we have $(1.01)_2$ left so $(010)_2$ is the fraction, because of hidden bit normalization! So, $(5)_{10} = (0 1001 010)_2$.
\begin{itemize}
	\item largest step before: $(0 1001 001)_2$
	\item smallest step after: $(0 1001 011)_2$
\end{itemize}

\subsubsection*{g. }
\begin{itemize}
    \item round down \newline
        So, we know the first bit is negative, thus we need to round up our magnitude so it rounds towards $- \infty$. Let's then focus on rounding $(10.1011)_2$ up. 
        $(10.1011)_2 = (1.01011)_2 * 2^1$
        then \newline - Exponent: $(1000)_2$ since bias is 7. \newline 
        - Fraction: $(011)_2$ since we round up \newline 
        Thus: $(1 1000 011)_2$ is the number in binary8.
    \item round up \newline
        Here, we must round down the magnitude since it is negative.

        $(1 1000 010)_2$
    \item round to zero\newline
        This would be $x_+$ since it is closer to 0 so $(1 1000 010)_2$
    \item round towards nearest\newline
        The nearest would be $(1 1000 011)_2$.
\end{itemize}

\subsubsection*{h. }

Lets try x smaller than subnormal, even if we might not be able to represent it. The smallest subnormal representable in binary8, as established before, is $2^{-6} * 2^{-3} = 2^{-9}$. We also established that the machine epsilon is 0.125. So, let's try with $x = 2^{-12}$.

$$\frac{|round(x)-x|}{|x|}>12 \epsilon$$
$$\frac{2^{-9}-2^{-12}}{2^{-12}}>1.5$$ assuming we round $2^{-12}$ to nearest subnormal so $2^{-9}$
$$7>1.5$$ which holds. So x = $2^{-12}$ works!
\vfill
\newpage

\subsection*{Question 3 }



\subsubsection*{a.} 
True. Since $x \oplus x = \textrm{round}(x + x) = \textrm{round}(2x) = 2x$ since x is a finite fpn.

\subsubsection*{b.} 

False. Let us pick $x = 1$ and $y = 2^{-4} = 0.0625$. We'll assume binary8.
$$x \ominus y = \textrm{x - y} = \textrm{round}(1 - 0.0625) = \textrm{round}(0.9375)  = (0 0111 000)_2 = 1$$ using round up.

Then, $$x \ominus x = -(y\ominus x) = -(\textrm{round}(0.0625 - 1))$$ Then, assuming round up again,

$$ = -(\textrm{round}(0.0625 - 1)) = -\textrm{round}(-0.9375) = (0 0000 111)_2 = 0.875$$

but $0.875 \neq 1$. Thus, this does not hold by counterexample.
\vfill
\newpage

\subsection*{Question 4 }


\subsubsection*{a.} $-\infty/0 = -\infty$ assuming positive 0.
\subsubsection*{b.} $\infty/(-\infty) = \mathrm{NaN}$
\subsubsection*{c.} $3* \mathrm{NaN} - \mathrm{NaN} = \mathrm{NaN}$
\subsubsection*{d.} $0/(- \mathrm{NaN}) = \mathrm{NaN}$
\vfill
\newpage

\end{document}
