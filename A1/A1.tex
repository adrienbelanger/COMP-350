\documentclass[12pt,a4paper]{article}

\setlength{\headheight}{15pt}
\usepackage[margin=1in]{geometry}
\usepackage{fancyhdr}
\usepackage{amssymb}
\usepackage{xcolor}

\pagestyle{fancy}
\fancyhf{}
\renewcommand{\headrulewidth}{0pt}
\fancyhead[L]{COMP 350 - A1 - Adrien Bélanger - 261118168}

\begin{document}

\subsection*{Question 1}

So, sign is - because of the first bit. Then, we flip the bits, add 1 and we add up the power of twos to get the number in decimal representing the magnitude. Combine with the negative symbol, and we have our number!
\\ 
1. Flip the bits
$$11111111111111111111111111101010 = 000000000000000000000000000010101$$
2. Add 1
$$00000000000000000000000000010101 + 1 = 00000000000000000000000000010110 $$
3. Calculate the magnitude 
$$2^4 + 2^2 + 2^1 = 22$$
4. Combine
$$-22$$
\newline
Thus, our answer is $-22$.
\vfill
\newpage

\subsection*{Question 2}

Suppose we have a binary8 format which follows IEEE standard. In this format the width of the exponent field is 4, and the width of the fraction field is 3.

\subsubsection*{a.}
The exponent bias should be 7.
\subsubsection*{b.}
The machine espilon is 

$$0 0111 001 - 0 0111 000 = \epsilon = 2^{-3} = 0.125$$

\subsubsection*{c.}

Largest: $0 1110 111 = 2^8 * (1 + 2^-1 + 2^-2 + 2^4) = 256 * 1.875 = 480$
\newline
Smallest: $0 0001 000 = 1 * 2^{-7} = 0.0078125$
\subsubsection*{d.}

No, not all integers. For example, 479 cannot be stored.
Proof: We can store 480, as seen above. However, the smallest step under would be 0 1110 110
which is $2^8 * (1 + 2^-1 + 2^-2)_2 = 256 * 1.75 = 448$
\newline
There are no steps in between. Thus, since 479 is an integer in between -480 and 480 and it cannot be represented, we conclude that there exist an integer which cannot exist in between the smallest and largest number in this system.

\subsubsection*{e. \textcolor{red}{INCORRECT - REVIEW}} 
largest: $0 0000 111 =2^{-7} * (2^{-2} + 2^{-4} + 2^{-8}) = 0.002471923828125 $
\newline
smallest: $0 0000 001 = 2^{-7} * 0.0078125 = {0.0000610352}$

\subsubsection*{f. \textcolor{red}{INCORRECT - REVIEW}}

What are the two floating point numbers (neither is equal to 5) closest to 5?

5 is 0 1000 so
\begin{itemize}
	\item largest step before: 0 1000 000 = 4
	\item smallest step after: 0 1000 100 = 6
\end{itemize}

\subsubsection*{g. \textcolor{red}{INCORRECT - REVIEW}}
Given number $$-(10.1011)_2$$
Round it to a binary8 number using the four rounding modes.

\begin{itemize}
    \item round down \newline
        So, we know the first bit is negative, thus we need to round up our magnitude so it rounds towards $- \infty$. Let's then focus on rounding 10.1011 up. 
        $$10.1011 = 1.01011 * 2^1$$
        then \newline Exponent: 1000 since bias is 7. \newline 
        Fraction: 011 since we round up \newline 
        Thus: $1 1000 011$ is the number in binary8.
    \item round up \newline
        Here, we must round down the magnitude since it is negative.

        1 1000 010
    \item round to zero\newline
        This would be $x_+$ since it is closer to 0 so 1 1000 010
    \item round towards nearest\newline
        The nearest would be $1 1000 011$.
\end{itemize}

\subsubsection*{h.}
 Find a real x
 in the range of subnormal numbers such that
$$\frac{|round(x)-x|}{|x|}>12 \epsilon$$
\vfill
\newpage

\subsection*{Question 3}

Are the following statements true or false? If a statement is true, give a proof and if it's false, give a counterexample. We assume no overflow occurs in the calculations and the rounding mode used can be any of the four rounding modes.

\subsubsection*{a.} 
If x is a finite floating point number, then $x \oplus x =2x$

\subsubsection*{b.} 
 If x and y are two finite floating point number, then $x\ominus y=-(y\ominus x)$.

\vfill
\newpage

\subsection*{Question 4}

What are the values of the expressions 
\subsubsection*{a.} $-\infty/0 = \mathrm{NaN}$ 
\subsubsection*{b.} $\infty/(-\infty) = \mathrm{NaN}$
\subsubsection*{c.} $3* \mathrm{NaN} - \mathrm{NaN} = \mathrm{NaN}$
\subsubsection*{d.} $0/(- \mathrm{NaN}) = \mathrm{NaN}$
\vfill
\newpage

\end{document}
